\documentclass[]{article}
\usepackage[english]{babel}

\usepackage{amsthm}
\usepackage{amsfonts}
\usepackage{amsmath}
\usepackage{amssymb}
\usepackage{xcolor}
\usepackage{pagecolor}
\usepackage{sectsty}
\definecolor{hotpink}{rgb}{1,0.52,0.73}
\definecolor{rad}{rgb}{1,0.4,.4}
\colorlet{dgrey}{white!5!black!95!}

\def\bx{\mathbf x}
\def\by{\mathbf y}
\def\naturals{\mathbb{N}}
\def\integers{\mathbb{Z}}
\def\rationals{\mathbb{Q}}
\def\algebraics{\mathbb{A}}
\def\reals{\mathbb{R}}
\def\powerset{\mathcal{P}}



\sectionfont{\color{hotpink}}  % sets colour of sections
\pagecolor{dgrey}
\color{white}

%%% qn/answer style
\theoremstyle{remark}
\newtheoremstyle{qnstyle}
  {1.5em} % Space above
  {} % Space below
  {} % Body font
  {} % Indent amount
  {\bfseries \color{rad}} % Theorem head font
  {.} % Punctuation after theorem head
  {.5em} % Space after theorem head
  {} % Theorem head spec (can be left empty, meaning `normal')

\theoremstyle{qnstyle}
\newtheorem{question}{Question}
\newtheoremstyle{answerstyle}
    {-.5em} % Space above
    {} % Space below
    {} % Body font
    {} % Indent amount
    {\bfseries \color{rad}} % Theorem head font
    {.} % Punctuation after theorem head
    {.5em} % Space after theorem head
    {} % Theorem head spec (can be left empty, meaning `normal')
\theoremstyle{answerstyle}
\newtheorem*{answer}{Answer}

\begin{document}
\noindent






%%%%%%%%%%%%%%%%%%%%%%%%%%%%%%%%%% LECTURE 2
\section* {Lecture 2}

\begin{question}
    Prove the contrapositive.
\end{question}
\begin{answer}
    Recall $A \Rightarrow B$ in terms of logical quantifiers and apply commutativity.
\end{answer}

\begin{question}
    State the Peano axioms defining $\mathbb{N}$.
\end{question}
\begin{answer}
    The natural numbers $\naturals$ is a set containing the element '1' 
    with an operation `+1' satisfying \\
    (i) $\forall n \in \naturals$, $n+1 \neq 1$; \\
    (ii) $\forall m,n \in \naturals$, if $m \neq n$, then $m+1 \neq n+1$; \\
    (iii) for any property $P(n)$, if $P(1)$ is true and $\forall n \in \naturals$,
    $P(n) \Rightarrow P(n+1)$, then $P(n)$ is true for all natural numbers.
\end{answer}

\begin{question}
    Define the operation `+k' for $k \in \naturals$ in terms of `+1'.
\end{question}
\begin{answer}
    For every natural number $n$, $n+(k+1) = (n+k)+1$.\\
    This is defined by induction on $P(k) =$ `` `$+k$' is defined''
\end{answer}





%%%%%%%%%%%%%%%%%%%%%%%%%%%%%%%%%% LECTURE 3
\section* {Lecture 3}

\begin{question}
    Show WPI and SPI are equivalent.
\end{question}
\begin{answer}
    To show WPI implies SPI, apply the former to $Q(n) = ``P(m)\; \forall\; m \leq n$''.
    To SPI implies WPI is self-evident.
\end{answer}

\begin{question}
    What is the well-ordering principle?
\end{question}
\begin{answer}
    If $P(n)$ holds for some $n \in \naturals$, then there is a least $n \in \naturals$ s.t. $P(n)$
 holds. 
\end{answer}

\begin{question}
    Prove that SPI is equivalent to WOP.
\end{question}
\begin{answer}
    Consider P(n) false and apply WOP and premises of SPI to obtain contradiction to show WOP implies SPI.
    To show SPI implies WOP, suppose no $n$ st. $P(n)$ and consider $Q(n) = \neg (P(n))$ with SPI.
\end{answer}





%%%%%%%%%%%%%%%%%%%%%%%%%%%%%%%%%% LECTURE 4
\section* {Lecture 4}

\begin{question}
    Define the highest common factor $c$ of $a, b \in \naturals$
\end{question}
\begin{answer}
    (i) $c|a$ and $c|b$ \\
    (ii) $d|a$ and $d|b \Rightarrow d|c$.
\end{answer}
\begin{question}
    Define the highest common factor $c$ of $a, b \in \naturals$
\end{question}
\begin{answer}
    (i) $c|a$ and $c|b$ \\
    (ii) $d|a$ and $d|b \Rightarrow d|c$.
\end{answer}





%%%%%%%%%%%%%%%%%%%%%%%%%%%%%%%%%% LECTURE 5
\section* {Lecture 5}

\begin{question}
    State Euclid's algorithm on $a,b \in \naturals$
\end{question}
\begin{answer}Note: $r_{i+1} < r_{i} < a$\\
    $a = q_1 b + r_1$\\
    $b = q_2 r_1 + r_2$\\
    $r_1 = q_3 r_2 + r_3$\\
    $r_{n-2} = q_n r_{n-1} + r_n$\\
    $r_{n-1} = q_{n+1} r_n + 0$\\
    Output: $r_n$
\end{answer}

\begin{question}
    Prove Euclid's algorithm returns the hcf of its input.
\end{question}
\begin{answer} 
    Prove properties by induction
\end{answer}

\begin{question}
    State Bezouts Theorem
\end{question}
\begin{answer} 
    Let $a, b \in \naturals$. Then the equation 
        $$ax + by = c$$ 
    has a solution in the integers iff $(a,b) | c$.
\end{answer}

\begin{question}
    Prove Bezouts Theorem with Euclid's algorithm
\end{question}
\begin{answer} 
    Run Euclid's algorithm with input $a,b$ to obtain an output $r_n$. At step n, we have
    $r_n = xr_{n-1} + yr_{n-2}$ for some $x, y \in \integers$. Continuing by induction 
    we have $\forall i = 2,\ldots, n-1, r_n = xr_ i+ yr_{i-1}$ for some $x,y \in \integers$.
    Thus $r_n = xa + yb$ for some $x,y \in \integers$ from step 1 and 2.
\end{answer}

\begin{question}
    Prove $\forall a,b \in \naturals,\; \exists x,y \in \integers$ s.t. $xa + yb = \text{hcf}(a,b)$ by minimality argument.
\end{question}
\begin{answer} 
    Let $h$ be the least positive linear integer of the form $xa + yb$ for some $x,y \in \integers$.\\
    Use divisibility and minimality to show it satisfies conditions for hcf.
\end{answer}

\begin{question}
    Prove if $p$ is a prime and $p | ab$, then $p|a$ or $p|b$.
\end{question}
\begin{answer} 
    Suppose $p \nmid a$ and show $p | b$ using Bezout.
\end{answer}

\begin{question}
    (Fundamental theorem of arithmetic) Prove every natural number $n \leq 2$ is
    expressible as a product of primes, uniquely up to ordering.
\end{question}
\begin{answer} 
    Factorisation can be shown by induction.\\
    For uniqueness, suppose two different factorisations, reorder, divide out with $p|ab$ lemma 
    and use induction.
\end{answer}





%%%%%%%%%%%%%%%%%%%%%%%%%%%%%%%%%% LECTURE 6
\section* {Lecture 6}






%%%%%%%%%%%%%%%%%%%%%%%%%%%%%%%%%% LECTURE 6
\section* {Lecture 7}

\begin{question}
    Prove inverses are unique modulo $n$.
\end{question}
\begin{answer} 
    Untangle definitions.
\end{answer}

\begin{question}
    Prove $a$ has an inverse modulo $n$ iff $(a,n) = 1$
\end{question}
\begin{answer} 
    Chain of equivalences using Bezout.
\end{answer}






%%%%%%%%%%%%%%%%%%%%%%%%%%%%%%%%%% LECTURE 8
\section* {Lecture 8}

\begin{question}
    State the Chinese Remainder Theorem
\end{question}
\begin{answer}
    Let $m,n$ be coprime and $a,b \in \integers$.
    Then there is a unique solution modulo $mn$ 
    to the simultaneous congruences\\
    $ equiv a \mod m$ and $x \equiv b \mod n$.
\end{answer}

\begin{question}
    Prove the Chinese Remainder Theorem
\end{question}
\begin{answer}
    Use Bezout $(m,n) = 1$ to construct an $x$ which satisfies conditions
    Show uniqueness by considering new solution $y$ taken $\mod m,n$ and combine 
    algebraically to show it is congruent $\mod mn$.
\end{answer}

\begin{question}
    State Fermats Little Theorem
\end{question}
\begin{answer}
    Let $p$ be prime. Then $a^p = a \mod p \forall a \in \integers$.
    Equivalently, $a^{p-1} \cong 1 \mod p \;\; \forall \; a \not\equiv 0 \mod p$
\end{answer}

\begin{question}
    Prove Fermats Little Theorem
\end{question}
\begin{answer}
    If $a \not\equiv 0 \mod p$, then $a$ is a unit $\mod p$.
    Hence the numbers $a, 2a, \ldots, (p-1)a$ are pairwise incongruent
    modulo $p$ and $\equiv 0 \mod p$, so they are $1,2,\ldots,p-1$ in some order.
    Hence 
        $$a \cdot 2a \cdot 3a \cdot \ldots \cdot (p-1)a \mod p = 1 \cdot 2 \ldots \cdot(p-1)$$
    or
        $$a^{p-1} (p-1)! \equiv (p-1)! \mod p$$
    But we can cancel it to obtain $a^{p-1} \equiv 1 \mod p$.
\end{answer}

\begin{question}
    State the Fermat-Euler Theorem
\end{question}
\begin{answer}
    Let $(a,m) = 1$. Then $a^{\phi(m)} \equiv 1 \mod m$.
\end{answer}

\begin{question}
    Prove the Fermat-Euler Theorem
\end{question}
\begin{answer}
    Consider the set of units modulo $m$ and apply same rearrangement argument as FLT.
\end{answer}





%%%%%%%%%%%%%%%%%%%%%%%%%%%%%%%%%% LECTURE 9
\section* {Lecture 9}

\begin{question}
    Let $p$ be a prime. Then $x^2 \equiv 1 \iff x \equiv +1 \mod p$ or $x \equiv -1 \mod p$. 
\end{question}
\begin{answer}
    Convert $p | ab$ lemma into a modular arithmetic statement and apply.
\end{answer}

\begin{question}
    State Wilson's theorem
\end{question}
\begin{answer}
    Let $p$ be prime. Then $(p-1)! \equiv -1 \mod p$
\end{answer}

\begin{question}
    Prove Wilson's theorem
\end{question}
\begin{answer}
    True for $p = 2$, so assume $p > 2$. Consider pairing up elements and recall
     $\pm 1$ are the only self-inverse.
\end{answer}

\begin{question}
    Let $p$ be an odd prime. Prove that then $-1$ is a square $\iff p \equiv 1 \mod 4$.
\end{question}
\begin{answer}
    For $p \equiv 1 \mod 4$, \\
    Apply Wilson's theorem  and manipulate to find explicit expression for 
    element which squares to $-1$.
    For $p \not \equiv 1 \mod 4$, \\
    Apply FLT to obtain $1 \equiv -1 \mod p$.
\end{answer}

\begin{question}
    Outline the RSA Scheme
\end{question}
\begin{answer}
    I think of two large primes, $p,q$.\\
    Let $n =pq$ and pick and {\it encoding exponent} $e$ coprime to 
    $\phi(n) = (p-1)(q-1)$.\\
    I publish the pair (n,e).\\ \\
    To send me a message (ie a sequence of numbers) you
    chop it into pieces/numbers $M < n$ and send me
    $M^e \mod n$, computed quickly by repeated squaring.\\ \\
    To decrypt, I work out $d$ st. $ed \equiv 1 \mod \phi(n)$.\\
    Then I compute $(M^e)^d = M^{k\phi(n) + 1}$ for some $k \in \integers$ \\
    =$M \mod n$ by Fermat-Euler.
\end{answer}





%%%%%%%%%%%%%%%%%%%%%%%%%%%%%%%%%% LECTURE 10
\section* {Lecture 10}

\begin{question}
    Prove (a level style) there is no rational $x$ with $x^2 = 2$.
\end{question}
\begin{answer}
    Suppose $x^2 = 2$. We may assume $x>0$ since $(-x)^2 = x^2$. If $x$ is rational,
    then $x = \frac{a}{b}$ for some $a,b \in \naturals$. 
    Thus $\frac{a^2}{b^2} =2$, or $a^2 = 2b^2$. But the exponent of 2 in the 
    prime factorisation is even while the exponent of 2 in $2b^2$ is odd,
    contradicting the FTA.
\end{answer}

\begin{question}
    Prove (constructively) there is no rational $x$ with $x^2 = 2$.
\end{question}
\begin{answer}
    Suppose $x^2 = 2$ for some $x = \frac{a}{b}$ with $a,b \in \naturals$. Then
    for any $c,d \in \integers$ $cx+d$ is of the form $\frac{e}{b}$ for some
    $e \in \integers$. Thus if $cx + d > 0$, then $cx + d > \frac{1}{b}$.\\
    Thus if $cx+d>0$, then $cx+d > \frac{1}{b}$.\\
    But $0 < x-1 < 1$ since $1 < x <2$.\\
    So if $n$ is sufficiently large,
        $$0 < (x-1)^n < \frac{1}{b}$$
    But for any $n \in \naturals$, $(x-1)^n$ is of the form $cx+d$ for some
    $c,d \in \integers$, since $x^2 = 2$. This is a contradiction.
\end{answer}

\begin{question}
    State the least upper bound axiom.
\end{question}
\begin{answer}
    Given any set S of reals that is non-empty and bounded above, S has
    a least upper bound.
\end{answer}





%%%%%%%%%%%%%%%%%%%%%%%%%%%%%%%%%% LECTURE 11
\section* {Lecture 11}

\begin{question}
    State the Axioms of Archimedes
\end{question}
\begin{answer}
    $\naturals$ is not bounded above in $\reals$. \\
    $\exists n in \naturals$ s.t. $nx > y \; \forall x,y \in \reals$.
\end{answer}

\begin{question}
    Prove the Axioms of Archimedes
\end{question}
\begin{answer}
    Suppose supremum of $\naturals$ existed and show a greater $n \in \naturals$ exist.
\end{answer}

\begin{question}
    Prove that for all $t > 0$, $\exists n \in \naturals$ with $\frac{1}{n} < t$.
\end{question}
\begin{answer}
    Given $t > 0$, there is an $n > \frac{1}{t}$ by Archamedian property, hence
    $\frac{1}{n} < t$.
\end{answer}

\begin{question}
    Prove that there exists $x \in \reals$ with $x^2 = 2$.
\end{question}
\begin{answer}
    Construct a set, and prove supremum satisfies $x^2=2$.
\end{answer}





%%%%%%%%%%%%%%%%%%%%%%%%%%%%%%%%%% LECTURE 12
\section* {Lecture 12}

\begin{question}
    What does it mean for the rationals to be dense in $\reals$?
\end{question}
\begin{answer}
    $\forall a < b \in \reals$, $\exists c \in \rationals$ with $a < c < b$.
\end{answer}

%% Give an intuitive account for this proof
\begin{question}
    Prove the rationals are dense in the reals
\end{question}
\begin{answer}
    We may assume that $a \leq 0$. \\ \\
    By Archimedian property, $\exists n \in \naturals$ with $\frac{1}{n} < b-a$.\\
    By the Axiom of Archimedes, $\exists N \in \naturals$ s.t. $N > b$.\\
    Let $T = \{k \in \naturals: \frac{k}{n} \leq b\}$\\
    then $Nn \in T$, so $T \neq \emptyset$.\\
    By WOP, T has a least element m. Set c = (m-1)/n.\\
    Since $m-1 \not \in T, c < b$.\\
    If $c \leq a$, then $\frac{m}{n} = c + \frac{1}{n} < a+b-a = b$ \\
    Which is a contradiction, hence $a < c < b$.
\end{answer}

\begin{question}
    When do we say that the sequence $a_1, a_2, a_3, \ldots$ tends to
    the limit $l \in \reals$?
\end{question}
\begin{answer}
    $\forall \epsilon > 0, \exists N \in \naturals$ s.t. $n \leq N$,
    $|a_n - l| < \epsilon$.
\end{answer}





%%%%%%%%%%%%%%%%%%%%%%%%%%%%%%%%%% LECTURE 13
\section* {Lecture 13}

\begin{question}
    Prove that every bounded monotonic sequence converges.
\end{question}
\begin{answer}
    Show that sequence converges to supremum.
\end{answer}

\begin{question}
    Prove that if $a_n \leq d \; \forall n$ and $a_n \to c$ as $n \to \infty$,
    $c \leq d$.
\end{question}
\begin{answer}
    Think about this geometrically for intuition, assume for contradiction.
\end{answer}





%%%%%%%%%%%%%%%%%%%%%%%%%%%%%%%%%% LECTURE 14
\section* {Lecture 14}

\begin{question}
    Does every $x$, $0 \leq x < 1$ have a decimal expansion?
\end{question}
\begin{answer}
    Construct a sequence of $x_k$ such that you can pick a maximal element to
    generate the decimal expansion of $x$.
\end{answer}

\begin{question}
    Prove that if a decimal is periodic, then it is rational.
\end{question}
\begin{answer}
    Find the rational expression for it
\end{answer}

\begin{question}
    Prove that if a decimal is rational, then it is periodic.
\end{question}
\begin{answer}
    
\end{answer}

\begin{question}
    Prove $e$ is irrational.
\end{question}
\begin{answer}
    Suppose $e = \frac{p}{q}$. This would mean $q!e$ is integral.\\
    Consider the infinite expansion ot show that $q!e = N + x$ where $N \in \naturals$,
    and $0 < x < 1$, by bounding with geo series. Contradiction.
\end{answer}





%%%%%%%%%%%%%%%%%%%%%%%%%%%%%%%%%% LECTURE 15
\section* {Lecture 15}

\begin{question}
    Prove that, for any polynomial $P$, $\exists$ constant $K$ such that
        $$|P(x) - P(y)| \leq K|x - y| \;\; \forall 0 \leq x,y \leq 1$$.
\end{question}
\begin{answer}
    Suppose
        $$P(x) = a_d x^d + a_{d-1} x^{d-1} + \cdots + a_1x + a_0$$
    Consider $P(x) - P(y)$ and factor out $(x-y)$ and bound to find a const.
\end{answer}

\begin{question}
    Prove that a non-zero polynomial of degree $d$ has at most $d$ roots.
\end{question}
\begin{answer}
    Induction on number of roots for polynomial of degree $d$, rewrite the polynomial
    as a product of new root and some polynomial $q(x)$ by long division.
\end{answer}

\begin{question}
    Prove the number 
        $$L = \sum_{n=1}^{\infty}{\frac{1}{10^{n!}}}$$
    is transcendental.
\end{question}
\begin{answer}
    Suppose $L$ is the root of a polynomial $P$. \\
    There exists a $K$ such that $|P(x) - P(y)| \leq K|x - y| \;\;
    \forall 0 \leq x,y \leq 1$. \\
    do it do it do it do it do it do it do it do it do it do it do it do it do it do it do it do it do it
\end{answer}





%%%%%%%%%%%%%%%%%%%%%%%%%%%%%%%%%% LECTURE 16
\section* {Lecture 16}

\begin{question}
    If A is a set and $P$ is a property of (some) elements of A, 
    how do we write the subset of A comprising of those elements 
    for which $P(x)$ holds?
\end{question}
\begin{answer}
    $${x \in A: P(x)}$$
\end{answer}

\begin{question}
    Write $A \backslash B$ in set notation
\end{question}
\begin{answer}
    $$\{ x \in A: x \not \in B\}$$
\end{answer}

\begin{question}
    If $A_1, A_2, A_3, \ldots$ are sets then what is 
        $$\cap_{n=1}^\infty A_n?$$
\end{question}
\begin{answer}
    $\{x: x \in A_n$ for all $n \in \naturals \}$
\end{answer}

\begin{question}
    Prove you cannot form $\{ x: P(x) \}$
\end{question}
\begin{answer}
    Construct $X = \{x: x$ is a set and $x \not \in x \}$,\\
    and consider whether $X \in X$
\end{answer}





%%%%%%%%%%%%%%%%%%%%%%%%%%%%%%%%%% LECTURE 17
\section* {Lecture 17}

\begin{question}
    Define the binomial coefficient $n \choose k$
\end{question}
\begin{answer}
    $${n \choose k} = \big \vert \{ S \subseteq \{1, 2, \ldots, n\}: |S| = k \}|$$
\end{answer}

\begin{question}
    State the inclusion-exclusion principle
\end{question}
\begin{answer}
    $$|S_1 \cup S_2 \cup \ldots \cup S_n| = \sum_{|A|=1}|S_a| 
        - \sum_{|A|=2}|S_A| + \sum_{|A|=3}|S_A| - \ldots
            + (-1)^{n+1}\sum_{|A|=n}|S_A|$$
where $S_A = \cap_{i\in A} S_i$
Equivalently,
    $$\big \vert \bigcap_{i = 1}^n S_i \big \vert = 
    \sum_{k=1}^n{(-1)^{k+1}} \mkern-20mu \sum_{\substack{A \subseteq \{1, 2, \ldots, n \} \\ |A| = k}}
        \big \vert \bigcap_{i \in A} S_i \big \vert.$$
\end{answer}

\begin{question}
    Prove the inclusion-exclusion principle.
\end{question}
\begin{answer}
    Suppose $x \in S_i$ $k$ times and prove that it is only counted once via counting.
\end{answer}





%%%%%%%%%%%%%%%%%%%%%%%%%%%%%%%%%% LECTURE 18
\section* {Lecture 18}

\begin{question}
    Formally define a function $f: A \to B$.
\end{question}
\begin{answer}
    A function from $A$ to $B$ is a subset $f \subseteq A \times B$ such that
    for all $x \in A$, there is a unique $y \in B$ such that $(x,y) \in f$.
\end{answer}

\begin{question}
    When do we say that a function $f: A \to B$ is injective?
\end{question}
\begin{answer}
    $\forall a, a' \in A$, \\
    $a \not = a' \Rightarrow f(a) \not = f(a')$, or equivalently \\
    $f(a) = f(a') \Rightarrow a = a'$.
\end{answer}

\begin{question}
    When do we say that a function $f: A \to B$ is surjective?
\end{question}
\begin{answer}
    If $\forall b \in B$, $\exists a \in A$ such that $f(a) = b$.
\end{answer}

\begin{question}
    What is the definition of the indicator function?
\end{question}
\begin{answer}
    \begin{align*}
        1_A & \mkern-5mu:  \quad x \to \{0, 1\} \\
        1_A & = \left\{
            \begin{array}{ll}
                1 & \quad x \in A \\
                0 & \quad x \not \in A
            \end{array}
        \right.
    \end{align*}
\end{answer}

\begin{question}
    When do we say that $f: A \to B$ is inverse
\end{question}
\begin{answer}
    If $\exists g: B \to A$ such that \\
    $g \circ f = \text{id}_A$ and $f \circ g = \text{id}_B$.
\end{answer}





%%%%%%%%%%%%%%%%%%%%%%%%%%%%%%%%%% LECTURE 19
\section* {Lecture 19}

\begin{question}
    Given $f: A \to B$, when is there a map $g: B \to A$ such that 
    $g \circ f = \text{id}_A$?
\end{question}
\begin{answer}
    If such a $g$ exists, using $a, a' \in A$ show $f$ injective. Conversely show that
    $f$ injective means that such a $g$ would exist.
\end{answer}

\begin{question}
    Given $f:A \to B$, when is it that a map $g: B \to A$ such that 
    $f \circ g = \text{id}_B$?
\end{question}
\begin{answer}
    We need $f(g(B)) = B$, so $f$ must be surjective. \\
    Conversely, if $f$ surjective show that such a $g$ exists.
\end{answer}

\begin{question}
    Show that $f: A \to B$ invertible $\iff$ $f$ is bijective.
\end{question}
\begin{answer}
    Consider both conditions on $f \circ g$ and $g \circ f$ and show together
    they imply bijectivity.
\end{answer}

\begin{question}
    Given $f: A \to B$, and $U \in B$, what is the pre-image of U, $f^{-1}(U)$?
\end{question}
\begin{answer}
    $$f^{-1}(U) = \{a \in A: f(a) \in U \}$$
\end{answer}

\begin{question}
    What is a relation on a set X?
\end{question}
\begin{answer}
    A subset $R \subseteq X \times X$, usually written $aRb$ for $(a,b) \in R$
\end{answer}

\begin{question}
    When is a relation $R$ reflexive?
\end{question}
\begin{answer}
    If $\forall x \in X$, $xRx$.
\end{answer}

\begin{question}
    When is a relation $R$ symmetric?
\end{question}
\begin{answer}
    If $\forall x,y \in X$, $xRy \Rightarrow yRx$.
\end{answer}

\begin{question}
    When is a relation R transitive?
\end{question}
\begin{answer}
    If $\forall x,y,z \in X$, $xRy$ and $yRz$ $\Rightarrow xRz$.
\end{answer}

\begin{question}
    Given a partition of X, define the equivalence relation R where equivalence
    classes are precisely the parts of the partition
\end{question}
\begin{answer}
    Define $a ~ b$ if $a$ and $b$ lie in the same part.
\end{answer}

\begin{question}
    Let $~$ be an equivalence relation on X. Prove the equivalence classes
    form a partition of X.
\end{question}
\begin{answer}
    Verify properties of a partition.
\end{answer}





%%%%%%%%%%%%%%%%%%%%%%%%%%%%%%%%%% LECTURE 20
\section* {Lecture 20}

\begin{question}
    Given an equivalence relation $R$ in a set $X$,
    define the quotient of $X$ by $R$.
\end{question}
\begin{answer}
    $X \backslash R = \{[x]: x \in X\}$
\end{answer}

\begin{question}
    Prove that any subset of $\naturals$ is countable.
\end{question}
\begin{answer}
    Apply WOP and remove element continually to produce a sequence $s_n$ of elements
    for the subset. Either finite or bijection which can be constructed by considering
    this sequence.
\end{answer}

\begin{question}
    Prove that
    (i) X is countable \\
    (ii) There is an injection $X \to \naturals$ \\
    are equivalent statements.
\end{question}
\begin{answer}
    (i) $\Rightarrow$ (ii) plain.\\
    (ii) implies bijection betwen $S = f(X)$, which is a subset of $\naturals$
    so there is bijection.
\end{answer}

\begin{question}
    Show that (iii) $X = \emptyset$ or there is a
    surjection $\naturals \to X$ implies (i) X is countable.
\end{question}
\begin{answer}
    If $X \not = \emptyset$ and there is a surjection
    $f: \naturals \to X$, define $g: X \to \naturals$ by
    $g(a) = \min f^{-1}(\{a\})$, which exists by WOP. g is injective, so $X$ is
    countable.
\end{answer}

\begin{question}
    Prove any subests of a countable set is countable.
\end{question}
\begin{answer}
    If $Y \subseteq X$ and $X$ is countable, then take the injection $X \to \naturals$ 
    restricted to $Y$.
\end{answer}





%%%%%%%%%%%%%%%%%%%%%%%%%%%%%%%%%% LECTURE 21
\section* {Lecture 21}

\begin{question}
    Prove that $\naturals \times \naturals$ is countable by counting over
    diagonals.
\end{question}
\begin{answer}
    Define $a_1 = (1,1)$ and $a_n$ inductively, by the sequence of coordinates that
    includes every point $(x,y) \in \naturals \times \naturals$ by counting through diagonals.
    Prove this is true by induction on $x+y$.
\end{answer}

\begin{question}
    Prove, algebraically, that $\naturals \times \naturals$ is countable.
\end{question}
\begin{answer}
    Define 
    \begin{align*}
        f\mkern-5mu: \quad & \naturals \times \naturals \to \naturals \\
                 & (x,y) \to 2^x 3^y
    \end{align*}
\end{answer}

\begin{question}
    Prove that a countable union of countable sets is countable.
\end{question}
\begin{answer}
    Given countable sets $A_1, A_2, A_3, \ldots$, we may list elements of $A_i$ by 
        $$a_1^{(i)}, a_2^{(i)}, a_3^{(i)}, \ldots$$ 
    Define 
    \begin{align*}
        f \mkern-5mu: \quad &\bigcup_{n \in \naturals} A_n \to \naturals \\
                 &x \to 2^i 3^j
    \end{align*}
    where $x = a_j^{(i)}$ for the least $i$ such that $x \in A_i$. \\
    This is an injection.
\end{answer}

\begin{question}
    Prove that $\reals$ are uncountable.
\end{question}
\begin{answer}
    Cantor's diagonal argument =)
\end{answer}






%%%%%%%%%%%%%%%%%%%%%%%%%%%%%%%%%% LECTURE 22
\section* {Lecture 22}

\begin{question}
    Prove there are uncountably many transcendental numbers.
\end{question}
\begin{answer}
    If $\reals \backslash \algebraics$ were countable, then since $\algebraics$ is countable,
    $\reals = \reals \backslash \algebraics \cup \algebraics$ would be countable. Contradiction.
\end{answer}

\begin{question}
    Prove for any set $X$, there is no bijection between $X$ and $\powerset(X)$
\end{question}
\begin{answer}
    Given $f: X \to \powerset(X)$,\\
    Let $S = \{x \in X: x \not \in f(x) \}$. $S$ does not belong to the image of $f$\\
    since $\forall x \in X$, $S$ and $f(x)$ differ in the element $x$ and thus $S \not = f(x)$,
    so $S$ is not mapped to.
\end{answer}

\begin{question}
    Let $\{A_i i \in I \}$ be a family of open intervals of $\reals$ which are
    pairwise disjoint. Prove the family is countable by considering rationals.
\end{question}
\begin{answer}
    Each interval $A_i$ contains a rational, and $\rationals$ is countable, so
    since the intervals are disjoint we have an injection from $I$ into $\rationals$.
    Hence the family $\{A_i:i \in I\}$ is countable.
\end{answer}

\begin{question}
    Let $\{A_i i \in I \}$ be a family of open intervals of $\reals$ which are
    pairwise disjoint. Prove the family is countable by considering length of intervals.
\end{question}
\begin{answer}
    The set $\{i \in I: A_i$ has length $\leq 1 \}$ is countable as it
    injects into $\frac{1}{2} \integers$.
    More generally, for each $n \in \naturals$, $\{i \in I: A_i$ has length $\leq \frac{1}{n} \}$
\\ \\
    Now $\{A_i: i \in I\}$ is countable as it is a countable union of
    countable sets.
\end{answer}





%%%%%%%%%%%%%%%%%%%%%%%%%%%%%%%%%% LECTURE 23
\section* {Lecture 23}

\begin{question}
    Given non-empty sets $A$ and $B$, $\exists$ injection $f: A \to B \iff \exists$
    surjection $g: B \to A$.  
\end{question}
\begin{answer}
    Construct a function $g$ which satisfies the desired condition.
\end{answer}

\begin{question}
    State the Schroder-Bernstein Theorem
\end{question}
\begin{answer}
    If $f: A \to B$ and $g: B \to A$ are injections, then $\exists$ bijection
    $h: A \to B$.
\end{answer}

\begin{question}
    Prove the Schroder-Bernstein Theorem.
\end{question}
\begin{answer}
    digest this and write it up
\end{answer}



\end{document}