\documentclass[]{article}
\usepackage[english]{babel}

\usepackage{amsthm}
\usepackage{amsfonts}
\usepackage{amsmath}
\usepackage{amssymb}
\usepackage{xcolor}
\usepackage{pagecolor}
\usepackage{sectsty}
\definecolor{hotpink}{rgb}{1,0.52,0.73}
\definecolor{rad}{rgb}{1,0.4,.4}
\colorlet{dgrey}{white!5!black!95!}

\def\bx{\mathbf x}
\def\by{\mathbf y}
\def\naturals{\mathbb{N}}
\def\integers{\mathbb{Z}}
\def\rationals{\mathbb{Q}}
\def\algebraics{\mathbb{A}}
\def\reals{\mathbb{R}}
\def\complex{\mathbb{C}}
\def\powerset{\mathcal{P}}
\def\mobius{\mathcal{M}}


\newenvironment{psmallmatrix}
  {\left(\begin{smallmatrix}}
  {\end{smallmatrix}\right)}


\sectionfont{\color{hotpink}}  % sets colour of sections
\pagecolor{dgrey}
\color{white}

%%% qn/answer style
\theoremstyle{remark}
\newtheoremstyle{qnstyle}
  {1.5em} % Space above
  {} % Space below
  {} % Body font
  {} % Indent amount
  {\bfseries \color{rad}} % Theorem head font
  {.} % Punctuation after theorem head
  {.5em} % Space after theorem head
  {} % Theorem head spec (can be left empty, meaning `normal')

\theoremstyle{qnstyle}
\newtheorem{question}{Question}
\newtheoremstyle{answerstyle}
    {-.5em} % Space above
    {} % Space below
    {} % Body font
    {} % Indent amount
    {\bfseries \color{rad}} % Theorem head font
    {.} % Punctuation after theorem head
    {.5em} % Space after theorem head
    {} % Theorem head spec (can be left empty, meaning `normal')
\theoremstyle{answerstyle}
\newtheorem*{answer}{Answer}

\begin{document}
\noindent


%opening
\title{}
\author{}

%%    Let $o(g) = m$, and $n \in \naturals$. Prove that





%%%%%%%%%%%%%%%%%%%%%%%%%%%%%%%%%% LECTURE 1
\section* {Lecture 1}

\begin{question}
    What must hold for (G, *) to be a group?
\end{question}
\begin{answer}
    (i) Identity exists \\
    (ii) Inverse exists\\
    (iii) (x*y)*z = x*(y*z)\\
\end{answer}

\begin{question}
    Sketch a proof for the uniqueness of inverses
\end{question}
\begin{answer}
    Suppose y,z are inverses of x. y = y*e = y*(x*z) and use associativity to show y = z. 

    Motivation: think about this fact is contingent on 
    - if associativity didn't hold, it'd make sense for there to exist 
    a left inverse and a right inverse. Therefore, we want to try and 
    sneak associativity into this. The only way we can do this is by 
    introducing three terms.
\end{answer}

\begin{question}
    Show that the identity element is unique
\end{question}
\begin{answer}
    Formally state what it means for $e$ and $\hat{e}$ to be identities and show that this implies they are equal.
\end{answer}







%%%%%%%%%%%%%%%%%%%%%%%%%%%%%%%%%% LECTURE 2
\section* {Lecture 2}

\begin{question}
    When is a group H cyclic?
\end{question}
\begin{answer}
    If there exists an element h in H such that each element of H is a power of h. 
\end{answer}

\begin{question}
    Prove the only subset of Z under addition is nZ
\end{question}
\begin{answer}
    If $H = \{0\} = 0\mathbb{Z}$

    Otherwise, choose $0 < n \in H$ with $n$ minimal. $n\mathbb{Z} \subseteq H$ by closure and inverses. We show $H = n\mathbb{Z}$. Suppose $\exists h \in H \setminus n \mathbb{Z}$, can write $h = nk + h'$, with $h' \in \{1,2,\ldots, n-1\}$

    But $h' = h - nk \in H$, contradicting minimality of n. Thus $H = n\mathbb{Z}$
\end{answer}

\begin{question}
    When is the function
        $$\theta: G \to H$$
    homomorphism from $(G, *_G)$ to $(H, *_H)$?
\end{question}
\begin{answer}
    $$\theta(x *_G y) = \theta(x) *_H \theta(y)$$
    $\forall x,y \in G$
\end{answer}

\begin{question}
    Let $\theta: G \to H$ be a homomorphism. What is the image of $\theta?$
\end{question}
\begin{answer}
    $\theta(G) = \{\theta(g) : g \in G\}$
\end{answer}

\begin{question}
    Prove the image of $\theta(G): G \to H$ is a subgroup of H
\end{question}
\begin{answer}
    Consider axioms
\end{answer}







%%%%%%%%%%%%%%%%%%%%%%%%%%%%%%%%%% LECTURE 3
\section* {Lecture 3}

\begin{question}
    Prove the composition of two homomorphisms is a homomorphism.
\end{question}
\begin{answer}
    Do it.
\end{answer}

\begin{question}
    Let $o(g) = m$, and $n \in \naturals$. Prove that
        $$g^n = e \iff m | n$$
\end{question}
\begin{answer}
    $\Leftarrow$ trivial verification \\
    $\Rightarrow$ Suppose $g^n = e$ and write $n = qm+r$, $0 \leq r < m$.
\end{answer}






%%%%%%%%%%%%%%%%%%%%%%%%%%%%%%%%%% LECTURE 4
\section* {Lecture 4}
\begin{question}
    How do we algebraically write $D_{2n}$?
\end{question}
\begin{answer}
    $$D_{2n} = <r,t \vert r^n =e, t^2 = e, trt = r^{-1}>$$
\end{answer}

\begin{question}
    Sketch a geometric proof that the group
    $D_{2n}$ is $\{e, r, r^2, \ldots, r^{n-1},t, rt, \ldots, r^{n-1}t \}$ 
    under composition.
\end{question}
\begin{answer}
    Let f be a symmetry of P
    f(1) maps to another vertex of P (say k).
    Let g be the rotation of f such that f(1) = 1.
    The vertex adjascent to 1 is either fixed or reflected.
    If it is fixed, g is the identity, meaning f is a rotation.
    Otherwise, g is a reflection then a rotation.
\end{answer}





%%%%%%%%%%%%%%%%%%%%%%%%%%%%%%%%%% LECTURE 5
\section* {Lecture 5}

\begin{question}
    How do we denote the set of all permutations of X?
\end{question}
\begin{answer}
    $\text{Sym}(X)$.
\end{answer}

\begin{question}
    What is the symmetric group of degree n?
\end{question}
\begin{answer}
    $$S_n = Sym(\{1, 2, \ldots, n \})$$
    the set of all permutations of $\{1, 2, \ldots, n\}$.
\end{answer}


\begin{question}
    Prove Sym(X) is a group under composition.
\end{question}
\begin{answer}
    *closure - composition of bijections forms a bijection \\ 
    *identity, define $c(x) = x$ for all $x$ in $X$ \\
    *inverse - $f \in Sym(X)$, since f is a bijection, $f^{-1}$ exists and is a bijection, satisfying $f^{-1} \circ f = e = f^{-1} \circ f$ \\
    *associativity - composition of functions is associative
\end{answer}


\begin{question}
    Define (informally) a k-cycle
\end{question}
\begin{answer}
    $\sigma = (a_1, a_2, \ldots, a_k)$,
mapping $a_1$ to $a_2$, $a_2$ to $a_3, \ldots, a_k$ to $a_1$. It leaves all other elements of $S_n$ fixed.
\end{answer}


\begin{question}
    Define (formally) a k-cycle
\end{question}
\begin{answer}
    Let $a_1, \ldots a_k$ be distinct integers in $\{1, \ldots n\}$. Suppose $\sigma \in S_n$ and

    $$\sigma(a_i) = a_{i+1}$$
    $$\sigma(a_k) = a_1$$

    for $1 \leq i \leq k-1$, and $\sigma(x) = x, \;\; \forall x \in \{1, \ldots, n \} / \{a_1, \ldots, a_n\}$.
\end{answer}

\begin{question}
    Show that disjoint cycles $\sigma, \tau$ commute
\end{question}
\begin{answer}
    Consider logically what happens and write it algebraically
\end{answer}






%%%%%%%%%%%%%%%%%%%%%%%%%%%%%%%%%% LECTURE 6
\section* {Lecture 6}


\begin{question}
    How do we write a permutation $\sigma$ as a product of disjoint cycles?
\end{question}
\begin{answer}
    $a_1 \in \{1, 2, \ldots, n \} = X$ \\
    Consider $a_1, \sigma(a_1), \sigma^2(a_1), \ldots$ \\
    Since $X$ is finite, $\exists$ minimal j s.t. $\sigma^j(a_i) \in \{a_1, \sigma(a_1), \ldots, \sigma^{j-1}(a_1) \}$ \\ \\

    Claim: $\sigma^j(a_1) = a_1$. \\
    Since if not, \\
    $\sigma^j(a_1) = \sigma^i(a_i)$ for $j > i \geq 1$ \\
    $\sigma^{j-i}(a_1) = a_1$ contradicting minimality of j. \\

\end{answer}

\begin{question}
    Show a k-cycle can be written as a product of transpositions
\end{question}
\begin{answer}
    $(a_1, a_2, \ldots, a_n) = (a_1, a_2)(a_2, a_3), \ldots(a_{k-2}, a_{k-1})(a_{k-1}, a_k)$.
\end{answer}

\begin{question}
    Let $\sigma, \tau$ be disjoint cycles in $S_n$. Prove $o(\sigma \tau) = \text{lcm}(o(\sigma), o(\tau))$
\end{question}
\begin{answer}
    ($\Leftarrow$)verify, recall $\sigma$ and $\tau$ commute because disjoint \\
    ($\Rightarrow$) Suppose $o(\tau \sigma) = n$,
    $$(\sigma \tau)^n = e$$
    $$\sigma^n \tau^n = e$$
    But $\sigma, \tau$ move different elements of X,\\
    So $\sigma^n = e$, $\tau^n = e$\\
    So $o(\sigma) \vert n$ and $o(\tau) \vert n$\\
    So $k = \text{lcm}(o(\sigma), o(\tau))$
\end{answer}


\begin{question}
    What does it mean that $sgn(S_n)$ is well defined?
\end{question}
\begin{answer}
    If $$\sigma = \tau_1 \ldots \tau_a$$
    $$= \tau_1' \ldots \tau_b'$$

    then $(-1)^a = (-1)^b$.
\end{answer}


\begin{question}
    Sketch a proof that the function

    sgn: $S_n \to \{\pm 1\}$
    $\sigma \to \text{sgn}(\sigma)$

    is well-defined.
\end{question}
\begin{answer}
    Given $\tau = (a,b)$ is a transposition, we show $\sigma \tau \mod 2 \cong \sigma + 1$  \\
    by considering the two cases: a and b are in the same cycle, and a and b are in distinct cycles.  \\
    In both cases we can rewrite the product of cycles such that the claim holds. \\
    We know c(identity) = n, so assuming $\sigma$ has two decompositions, we can show a = b mod 2, from which claim holds.
\end{answer}


\begin{question}
    Suppose (k,l) are in the same cycle within $\sigma$. How do we "factor" $\sigma (k,l)$ into disjoint cycles?
\end{question}
\begin{answer}
    $$(k, a_1, \ldots, a_r,l, b_1, \ldots, b_s)(k,l)$$
    $$(k, b_1, \ldots, b_s)(l, a_1, \ldots, a_r)$$
\end{answer}

\begin{question}
    Suppose (k,l) are in different cycles of $\sigma$. How do we "factor" $\sigma (k,l)$ into disjoint cycles?
\end{question}
\begin{answer}
    $$(k, a_1, \ldots, a_r)(l, b_1, \ldots, b_s)(k, l)$$
    $$= (k, b_1, b_2, \ldots, b_s, l_1, a_2, \ldots, a_r)$$
\end{answer}






%%%%%%%%%%%%%%%%%%%%%%%%%%%%%%%%%% LECTURE 7
\section* {Lecture 7}

\begin{question}
    Sketch a proof of the claim that the map

        $$\text{sgn}(S_n, \circ) \to (\{ \pm 1\}, \times)$$

    is a well-defined and non-trivial homomorphism. (assuming sgn well defined)
\end{question}
\begin{answer}
    - Well defined since $sgn(S_n)$ is well defined.
    - $sgn((1,2)) = -1$, so non-trivial. (not just equal to identity)
\end{answer}

\begin{question}
    Sketch a proof of the claim that the map
    $$\text{sgn}(S_n, \circ) \to (\{ \pm 1\}, \times)$$
    is a homomorphism (assuming well-defined and non-trivial)
\end{question}
\begin{answer}
    Suppose $a,b \in S_n$
    Writing out a,b as transpositions it follows that sgn(ab) = sgn(a)sgn(b).
\end{answer}

\begin{question}
    Let $H \leq G$, $a,b \in G$. Prove\\
    $a^{-1}b \in H \Rightarrow aH = bH$
\end{question}
\begin{answer}
    Suppose $a^{-1}b = k \in H$\\
    $b = ak \in aH$\\
    also $b \in bH$\\
    So $aH = bH$.
\end{answer}

\begin{question}
    What is a coset of G?
\end{question}
\begin{answer}
    Let $H \leq G$, and $g \in G$. The left coset gH is defined to be
    $$\{gh: h \in H\}$$, similar for right coset.
\end{answer}


\begin{question}
    What is the index of H in G denoted $\vert G:H \vert$?
\end{question}
\begin{answer}
    It is the number of left cosets of H in G.
\end{answer}

\begin{question}
    What is $|G:H|$ for finite G,H in terms of the order of G and H?
\end{question}
\begin{answer}
    $|G:H| = |G|/|H|$.
\end{answer}

\begin{question}
    Prove that if $aH \cap bH$ isn't empty, then $aH = bH$.
\end{question}
\begin{answer}
    Let $c$ be an element in the intersection.
    $c = ak$ for some $k$ in $H$, by writing set definition of $cH$ we see that $cH$ is a improper subset of $aH$.\\ \\
    Similarly, $a = ck^{-1}$, which is an element of $cH$ \\
    So $aH$ is a improper subset of $cH$. \\
    So $aH = cH$. Similarly $cH = bH$ so $aH = bH$.
\end{answer}


\begin{question}
    Let $H \leq G$, $a,b \in G$. Prove
    $aH = bH \rightarrow a^{-1}b \in H$
\end{question}
\begin{answer}
    $b \in bH = aH$
    $b = ah$ for some $h \ in H$
    $a^{-1}b = h \in H$
\end{answer}

\begin{question}
    What does Lagrange's Theorem state?
\end{question}
\begin{answer}
    Let H be a subgroup of the finite group G. Then the order of H divides the order of G.
\end{answer}

\begin{question}
    Prove Lagrange's theorem.
\end{question}
\begin{answer}
    G is partitioned into k distinct cosets of H. Each coset is the same size. So $\vert G\vert$ = $\vert H\vert$ k.
\end{answer}

\begin{question}
    Sketch a proof of Lagranges corollary.
\end{question}
\begin{answer}
    Apply Lagranges theorem to the subgroup $\rangle g \langle$ of G.
\end{answer}

\begin{question}
    State Lagranges corollary
\end{question}
\begin{answer}
    G is a finite group and $g \in G$. Then $o(g) \big \vert \vert G \vert$.
     In particular, $g^{\vert G \vert} = e$.
\end{answer}







%%%%%%%%%%%%%%%%%%%%%%%%%%%%%%%%%% LECTURE 8
\section* {Lecture 8}


\begin{question}
    When is a subgroup K called normal?
\end{question}
\begin{answer}
    if $gK = Kg$ for all $g \in G$.
\end{answer}


\begin{question}
    State the Fermat-Euler Theorem
\end{question}
\begin{answer}
    Let $a \in \mathbb{N}, n \in \mathbb{Z}, (a,n) = 1.$
    $$a^{\phi(n)}  \equiv 1 \mod n$$
\end{answer}


\begin{question}
    Prove that the inverse of $a \in R^*_n$ is in $R^*_n$
\end{question}
\begin{answer}
    Claim follows from Bezout on (a,n)=1 and rearranging.
\end{answer}






%%%%%%%%%%%%%%%%%%%%%%%%%%%%%%%%%% LECTURE 9
\section* {Lecture 9}

\begin{question}
    What is the first isomorphism theorem?
\end{question}
\begin{answer}
    Let G, H be groups and $$\theta: G \to H$$ a group homomorphism.
    Then $\text{Im} \theta \leq H$, $\text{Ker} \theta \trianglelefteq G$ \\
    and $G \backslash \text{Ker} \theta \cong \text{Im} \theta$
\end{answer}

\begin{question}
    (Sketch) proof that $K \trianglelefteq G$ means $G \backslash K$ exists
\end{question}
\begin{answer}
    Check coset multiplication is well defined, ie. that two cosets multiplied always give
    the same result (this $\iff$ normal). Verify closure, identity and inverse.
\end{answer}


\begin{question}
    Prove the 1st isomorphism theorem
\end{question}
\begin{answer}
    Construct an isomorphism $\phi$
    $$G \backslash \text{Ker}\theta \to \text{Im} \theta$$

    Let $K = \text{Ker}\theta,$ $$gK \to \theta(g)$$
    Prove $\phi$ is well-defined, homomorphism and bijection.
\end{answer}






\section*{Lecture 10}

\begin{question}
    What is a simple group?
\end{question}
\begin{answer}
    A group with no non-trivial normal subgroups.
\end{answer}

\begin{question}
    Prove that a homomorphism $\theta: G \to H$ is injective iff
        $$\text{Ker} \theta = \{ e \}$$
\end{question}
\begin{answer}
    Consider both directions to show it is injective.
\end{answer}


\begin{question}
    (i) Let $N \trianglelefteq G$, $H \leq G$. Prove that then $NH \leq G$. \\
    (ii) Let $N \trianglelefteq G$, $M \trianglelefteq G$, prove then $NM \trianglelefteq G$.
\end{question}
\begin{answer}
    diy
\end{answer}








\section* {Lecture 11}


\begin{question}
    Let $G$ be a group with subgroups $H$ and $K$. 
\end{question}
\begin{answer}
    Consider both directions to show it is injective.
\end{answer}

\begin{question}
    Prove that if, \\
    (i) each elem of G can be written as hk ($h \in H$, $k \in K$)\\
    (ii) $H \cap K = \{ e\}$\\
    (iii) $hk = kh \forall k \in K, h \in H$, \\
    then \\
    $G \cong H \times K$
\end{question}
\begin{answer}
    Show the map $$\theta: H \times K \to G$$
    $$\theta (h,k) \to hk$$
    is an isomorphism. 
\end{answer}

\begin{question}
    There are alternate, equivalent definitions of internal direct product. \\
    Show that \\
    (i) each elem of G can be written as hk ($h \in H$, $k \in K$)\\
    (ii) $H \cap K = \{ e\}$\\
    (iii) $hk = kh \forall k \in K, h \in H$, \\
    is equivalent to
    (i)' $H \trianglelefteq G, K \trianglelefteq G$
    (ii)' $H \cap K = \{ e\}$\\
    (iii) HK = G.
\end{question}
\begin{answer}
    $\Rightarrow$ use (iii) to show (i)' and (i) implies (iii)'\\
    $\Leftarrow$ Use (i)' to show (iii), (i) immediate from (iii)
\end{answer}






\section* {Lecture 12}

\begin{question}
    Classify groups of order 6 up to isomorphism.
\end{question}
\begin{answer}
    Consider order of elements with lagrange. if $o(g) = 6$, then $G \cong C_6$.\\
    Can't have all elems order 2, since $|G|$ isn't a power of two.\\
    We always have an element $a$ of order 3. ($g^2$ if o(g) = 6).\\
    Consider $b \in G \backslash \langle a \rangle$; $b^2 \in \langle a \rangle$ \\
    $bab^{-1} \in \langle a \rangle$ and consider cases.
\end{answer}


\begin{question}
    Classify groups of order 8 up to isomorphism
\end{question}
\begin{answer}
    If all elems order 2, then we have $C_2 \times C_2 \times C_2$. Prove by quotienting out\\
    If an elem has order 8, then $C_8$.\\
    Let $a$ have order 4. Consider $b \in G \backslash \langle a \rangle$.\\
    $b^2 \in \langle a \rangle$\\
    If $b^2 = a$ or $a^3$, then $o(b) = 8$, so $C_8$.\\
    Consider $bab^{-1} = a^{i}$, and evaluate $b^2 a b^{-2}$ in two ways to show $i = \pm 1$.\\
    Consider the four cases in $i$ and $b^2$ to get groups.
\end{answer}








\section*{Lecture 13}

\begin{question}
    Define a group action $\phi$.
\end{question}
\begin{answer}
    G acts on X with the map\\
    $\phi: G \times X \to X$\\
    $(g, x) \to \phi(g,x) = g(x)$ \\
    such that \\
    (1) closure (implied by notation)
    (2) gh(x) = g(h(x))
    (3) identity acting on x is sent to x
\end{answer}

\section*{Lecture 14}

\begin{question}
    Define a group action in terms of Sym(X)
\end{question}
\begin{answer}
    If G is a group and X is a set such that $$\Phi: G \to \text{Sym}(X)$$ 
    is a group homomorphism, then
    $$\phi: G \times X \to X$$
    $$(g, x) \to \phi_g(x)$$
    where $\Phi(g) = \phi_g$, is a group action.
\end{answer}

\begin{question}
    What is a faithful action?
\end{question}
\begin{answer}
    $\text{Ker}\Phi = \{e\}$; the only $x$ acted to the identity is the identity.
\end{answer}

\begin{question}
    What does Cayley's theorem state?
\end{question}
\begin{answer}
    Every group is isomorphic to a subgroup of Sym(X) for some X.
\end{answer}

\begin{question}
    Prove Cayley's theorem
\end{question}
\begin{answer}
    Consider the left regular action.
    $$\Phi: G \times G \to G$$
    $$(g,h) \to gh$$
    This action is faithful (show), so we have an injective homomorphism
    $$\Phi: G \to \text{Sym}(G)$$
    So $G \cong \text{Im}\Phi \leq \text{Sym}(G)$ as required.
\end{answer}

\begin{question}
    Define the orbit of $x \in X$
\end{question}
\begin{answer}
    $$\text{Orb}_G(x) = \{g(x): g \in G \} \subseteq X$$
    The orbit of $x$ is the set of points in X that it can map to.
\end{answer}

\section*{Lecture 15}

\begin{question}
    Define a transitive action
\end{question}
\begin{answer}
    G acts transitively on X if for any $x \in X$, $\text{Orb}_G(x) = X$.
\end{answer}

\begin{question}
    Prove the distinct G-orbits form a partition of X.
\end{question}
\begin{answer}
    Suppose $z \in \text{Orb}_G(x) \cap \text{Orb}_G(y)$.\\
    Show $ \text{Orb}_G(x) = \text{Orb}_G(z) = \text{Orb}_G(y)$.
\end{answer}

\begin{question}
    Show that $\text{Orb}_G(x)$ is invariant, i.e. 
    $$g(\text{Orb}_G(x)) \subseteq \text{Orb}_G(x).$$
\end{question}
\begin{answer}
    just do it
\end{answer}

\begin{question}
    Show that G is transitive on $\text{Orb}_G(x)$
\end{question}
\begin{answer}
    this means any element in $\text{Orb}_G(x)$ can go to any other element in $\text{Orb}_G(x)$. construct this
\end{answer}

\begin{question}
    Define the stabilizer of x in G
\end{question}
\begin{answer}
        $\text{Stab}_G(x) = \{g \in G: g(x) = x\}$
\end{answer}  

\begin{question}
    Prove $\text{Stab}_G(x)$ is a subgroup of G.
\end{question}
\begin{answer}
    just do it
\end{answer}  

\begin{question}
    Prove the orbit-stabilizer theorem.
\end{question}
\begin{answer}
    We prove that the index of $\text{Stab}_G(x)$ is $| \text{Orb}_G(x)|$. \\
    Consider the map
    $$\theta: \text{Orb}_G(x) \to (G:\text{Stab}_G(x))$$
    $$g(x) \to g \text{Stab}_G(x)$$
    Now prove this is a well-defined bijection.
\end{answer}  


\section* {Lecture 16}
miao
\section* {Lecture 17}

\begin{question}
    State Cauchy's theorem.
\end{question}
\begin{answer}
    Let G be a finite group and $p$ a prime that divides $|G|$. Then there 
    exists an element in $G$ of order $p$.
\end{answer}  

\begin{question}
    Prove Cauchy's theorem.
\end{question}
\begin{answer}
    $$X = \{ (x_1, x_2, \ldots, x_p): x_1 x_2 \cdots x_p = e, x_i \in G \}$$
    Let $H = \lbrace h: h^p = e \rbrace \cong C_p$
    which acts on X as follows
    $$H \times X \to X$$
    $$(h, (x_1,\ldots , x_p)) \to (x_2, x_3, \ldots, x_p, x_1)$$
    In general $$(h^i, (x_1,x_2, \ldots, x_p)) \to (x_{1+i}, x_{2+i}, \ldots, x_{p+i})$$
    With suffix $k$ taken $(k-1 \mod p) +1$. \\
    Check this is a group action. Using the fact $|\text{Orb}_H(\bar{x})| = 1$ or $p$ (by O/S),
    that $$p \; \Big\vert\; \sum_{\text{distinct orbits}}{\text{Orb}_H}(\bar{x}),$$ and that 
    $$\text{Orb}_H(e,e, \ldots e)) = 1$$ to show that 
    $\bar{x} \neq e$ exists such that $$\text{Orb}_H(\bar{x}) = 1, \; \bar{x} = (x, x, \ldots x)$$
    meaning $x^n = e$.
\end{answer}  

\begin{question}
    What is the conjugacy action?
\end{question}
\begin{answer}
    $$G \times G \to G$$
    $$(g, h) \to ghg^{-1}$$
\end{answer}

\begin{question}
    Given $z \in Z(G)$, what is $\text{ccl}_G(z)$?
\end{question}
\begin{answer}
    1
\end{answer}

\begin{question}
    Write $Z(G)$ in terms of the centralizers in $G$.
\end{question}
\begin{answer}
    $Z(G) = \bigcap_{h \in G} C_G(h)$
\end{answer}

\begin{question}
    How can we relate the order of two elements in the same conjugacy class
\end{question}
\begin{answer}
    They are equal
\end{answer}

\begin{question}
    What is the orbit and stabilizer of the conjugacy action?
\end{question}
\begin{answer}
    The orbit are conjugacy classes and the stabilizers are called centralizers.
\end{answer}

\begin{question}
    Prove that if $|G| = p^n$ for some prime $p$, then $Z(G)$ is non-trivial.
\end{question}
\begin{answer}
    $$G = \bigcup_{\text{distinct conj classes}} \text{ccl}_G(x)$$
    By Orbit-Stabilizer theorem, $\text{ccl}_G(x) \; \big \vert \; |G| = p^n$ \\
    So $|\text{ccl}_G(x)| = 1$ or $p\; \big \vert \; |\text{ccl}_G(x)|$\\
    But $$|G| = \sum_{x \in Z(G)}{\text{ccl}_G(x)} 
        + \sum_{\text{distinct classes w/} p\big\vert\text{ccl}_G(x)} {\text{ccl}_G(x)}$$
    So $$p\; \big \vert\; \sum_{x \in Z(G)}{\text{ccl}_G(x)} = Z(G)$$ so $|Z(G)| > 1$.
\end{answer}

\begin{question}
    Prove that if $G \big\backslash Z(G)$ is cyclic, then G is abelian.
\end{question}
\begin{answer}
    just do it lolololol
\end{answer}








\section* {Lecture 18}

\begin{question}
    Prove that if $G \big\backslash Z(G)$ is cyclic, then G is abelian.
\end{question}
\begin{answer}
    just do it lolololol
\end{answer}

\begin{question}
    Let $\sigma \in S_n$. Define the cycle type of $\sigma$.
\end{question}
\begin{answer}
    $\sigma$ can be written as a product of disjoint cycles including 1-cycles. Then the cycle type of $\sigma$ is 
    $(n_1, n_2, \ldots, n_k)$ where $n_1 \leq n_2 \cdots \leq n_k$ and the $i$th cycle of $\sigma$ has length $n_i$
\end{answer}

\begin{question}
    Prove that $\pi$ and $\sigma$ are conjugate in $S_n$ iff they have the same cycle type.
\end{question}
\begin{answer}
    $\sigma$ can be written as a product of disjoint cycles including 1-cycles. Then the cycle type of $\sigma$ is 
    $(n_1, n_2, \ldots, n_k)$ where $n_1 \leq n_2 \cdots \leq n_k$ and the $i$th cycle of $\sigma$ has length $n_i$. \\
    Consider $\tau$ conjugating $\sigma$, and consider the (general!) element $\sigma(a_{ij})$ and unwrap notation until
    you can write a product of disjoint cycles for the conjugation. \\ \\
    To show the converse, write $\pi$ similarly to $\sigma$ but with $b_{ij}$ and then show, with $\tau(a_{ij}) = b_{ij}$, that $\pi = \tau \sigma \tau^{-1}$.
\end{answer}

\begin{question}
    Let $x \in A_n$.  If $\text{C}_{A_n}(x) = \text{C}_{S_n}(x)$, how can we relate
    the size of $\text{cl}_{A_n}(x)$ and $\text{ccl}_{S_n}(x)$?
\end{question}
\begin{answer}
    $$|\text{ccl}_{A_n}(x)| = \frac{|\text{ccl}_{S_n}(x)|}{2}$$
\end{answer}

\begin{question}
    Let $x \in A_n$.  If $\text{C}_{A_n}(x) \nleq  \text{C}_{S_n}(x)$, how can we relate
    the size of $\text{cl}_{A_n}(x)$ and $\text{ccl}_{S_n}(x)$?
\end{question}
\begin{answer}
    $$|\text{ccl}_{A_n}(x)| = {|\text{ccl}_{S_n}(x)|}$$
\end{answer}






\section* {Lecture 19}

\begin{question}
    What are the number of elements in $S_n$ with $k_L$ cycles of length L?
\end{question}
\begin{answer}
    $$\frac{n!}{\prod_L{k_L! L^{k_L}}}$$
\end{answer}

\begin{question}
    Prove $A_5$ is a simple group.
\end{question}
\begin{answer}
    Suppose $N \trianglelefteq A_5$. Then $N$ is a union of conjugacy classes. \\
    $\Rightarrow |N| = 1 + 15a + 20b + 12c$ where $a,b \in \{0, 1\}, c \in \{0,1,2\}$. \\
    But by Lagrange, $|N| \big \vert |A_5| = 60$.\\
    Only possibility is $|N| = 1$ or $60$.
\end{answer}

\begin{question}
    Prove $\text{GL}_n(\mathbb{R})$ is a group under matrix multiplication.
\end{question}
\begin{answer}
    closure, identity, inverse: don't be noob
    associative: use index notation
\end{answer}

\begin{question}
    Prove $$\text{Det}: \text{GL}_n(\mathbb{R}) \to (\mathbb{R} \backslash \{0\}, \times)$$
        $$A \to \text{det}A$$
    is a surjective homomorphism
\end{question}
\begin{answer}
    do it
\end{answer}

\begin{question}
    What is $\mathbb{F}_p$?
\end{question}
\begin{answer}
    It is the finite field $$\mathbb{F}_p = (\{0,1,\ldots,p-1\}, +_p, \times_p).$$
\end{answer}








\section* {Lecture 20}

\begin{question}
    What is $|GL_3(\mathbb{F}_p)|$?
\end{question}
\begin{answer}
    Consider no. choices for each column. \\
    Column 1: $p^3 - 1$ \\
    Column 2: $p^3 - p$\\
    Column 3: $p^3 - p^2$
\end{answer}

\begin{question}
    What is $O(\mathbb{R})$
\end{question}
\begin{answer}
    Orthonormal group. $$\text{O}_n(\mathbb{R}) = \{A \in M_n: A A^T = I\}$$
\end{answer}








\section* {Lecture 21}

\begin{question}
    Prove $O(\mathbb{R})$ is a group
\end{question}
\begin{answer}
    :3
\end{answer}

\begin{question}
    Let $A \in \text{O}_n(\mathbb{R})$ and $\bx, \by \in M_n$, then \\
    (i) $A\bx \cdot A\by = \bx \cdot \by$\\
    (ii) $|Ax| = |x|$
    So A is an isometry.
\end{question}
\begin{answer}
    Start from givens and unravel definitions.
\end{answer}

\begin{question}
    Find all 2x2 orthogonal matrices.
\end{question}
\begin{answer}
    Use $AA^T = I$ definition to find conditions on the entries, then consider cases.
\end{answer}

\begin{question}
    Let $A \in \text{SO}_3(\reals)$. Show that then $A$ is conjugate to a matrix
    of the form

        $$\begin{pmatrix}
            \cos\theta & -\sin\theta & 0 \\
            \sin\theta &  \cos\theta & 0 \\
            0          &  0          & 1
        \end{pmatrix}$$

    for some $\theta \in [0, 2\pi).$ 

\end{question}
\begin{answer}
    Can show matrix is conjugate to a 3x3 matrix that fixes $e_3$.
    Then we just need to consider the 2x2 matrix case, from which it immediately
    follows 2x2 case.
\end{answer}

\begin{question}
    Write $\text{O}_3(\reals)$ in terms of $\text{SO}_3(\reals)$
\end{question}
\begin{answer}
    $\text{O}_3(\reals) = \text{SO}_3(\reals) \cup 
    \begin{psmallmatrix}
        -1 & 0 & 0 \\
         0 & 1 & 0 \\
         0 & 0 & 1
    \end{psmallmatrix} \text{SO}_3(\reals) $
\end{answer}






%%%%%%%%%%%%%%%%%%%%%%%%%%%%%%%%%% LECTURE 22
\section* {Lecture 22}

\begin{question}
    What is a Mobius transform?
\end{question}
\begin{answer}
    A Mobius transform is a function of a complex variable that can be written
        $$f(z) = \frac{az + b}{cz + d}$$
    for some $a,b,c,d \in \complex$ with $ad - bc \not = 0$.
\end{answer}

\begin{question}
    Prove any element of $O_3(\reals)$ is a product of at most
    3 reflections
\end{question}
\begin{answer}
    Can consider fixing an axis at at time by adding a new reflection for each.
\end{answer}

\begin{question}
    Prove the set $\mobius$ of all Mobius maps on $\complex$ is a group under 
    composition.
\end{question}
\begin{answer}
    Composition of maps is associative.\\
    $I(z) = z \in \mobius$. \\
    Closure can be verified manually by considering cases and bash.
\end{answer}

\begin{question}
    Prove 
        $$\text{GL}_2 (\complex)\backslash Z \cong \mobius$$
    where $Z = \{ \begin{psmallmatrix}
        \lambda & 0 \\
        0 & \lambda
    \end{psmallmatrix}: \lambda \in \complex / \{0\}\}$.
\end{question}
\begin{answer}
    Construct surjective homomorphism
    \begin{align*}
        \Phi: \quad & \text{GL}_2(\complex) \; \to \mobius \\
        & \begin{psmallmatrix} 
            a & b \\ 
            c & d 
        \end{psmallmatrix} \quad  \to \frac{az+b}{cz+d}
    \end{align*}
    and apply first isomorphism theorem.
\end{answer}






%%%%%%%%%%%%%%%%%%%%%%%%%%%%%%%%%% LECTURE 23
\section* {Lecture 23}

\begin{question}
    Prove 
        $$\text{SL}_2 (\complex)\backslash \{\pm I\} \cong \mobius$$
\end{question}
\begin{answer}
    Construct surjective homomorphism
    \begin{align*}
        \Phi: \quad & \text{SL}_2(\complex) \; \to \mobius \\
        & \begin{psmallmatrix} 
            a & b \\ 
            c & d 
        \end{psmallmatrix} \quad  \to \frac{az+b}{cz+d}
    \end{align*}
    and apply first isomorphism theorem.
\end{answer}

\begin{question}
    Prove every Mobius map can be written as a composiiton of maps of the following
    forms: \\
    (i) $z \to az$, $a \not = 0$ \\
    (ii) $z \to z + b$ \\
    (iii) $z \to \frac{1}{z}$
\end{question}
\begin{answer}
    Let $f(z) = \frac{az+b}{cz+d}$
    Consider $c = 0$ and $c \not = 0$ cases and construct $f$ by composition.
\end{answer}

\begin{question}
    What does it mean for a group $G$ to act triply transitively on a set $X$?
\end{question}
\begin{answer}
    Given distinct $x_1, x_2, x_3 \in X$ and distinct $y_1, y_2, y_3 \in X$, 
    then there exists a $g \in G$ such that $g(x_i) = y_i$.
\end{answer}

\begin{question}
    Prove the action  of $\mobius$ on $\complex_\infty$ is sharply triply transitive.
\end{question}
\begin{answer}
    Label first triple $\{z_0, z_1, z_\infty \}$ and second triple $\{w_0, w_1, w_\infty\}$.\\
    We construct $g \in \mobius$ such that
    \begin{align*}
        g: \quad & z_0 \to 0 \\
                 & z_1 \to 1 \\
                 & z_\infty \to \infty \
    \end{align*}

    First suppose $z_1, z_2, z_3 \not = \infty$. Then
        $$g(z) = \frac{(z-z_0)(z_1 - z_\infty)}{(z-z_\infty)(z_1 - z_0)}$$
    Check: $``ad-bc'' = (z_0 - z_\infty)(z_1 - z_\infty)(z_1 - z_0) \not = 0$. \\
    If $z_\infty = \infty$, then $g(z) = \frac{z-z_0}{z_1-z_0}$\\
    If $z_1 = \infty$, then $g(z) = \frac{z-z_0}{z - z_\infty}$\\
    If $z_\infty = \infty$, then $g(z) = \frac{z_1 - z_\infty}{z - z_\infty}$. \\
    \\
    Similarly find $h$ such that 
    \begin{align*}
        h: \quad & z_0 \to 0 \\
                 & z_1 \to 1 \\
                 & z_\infty \to \infty \
    \end{align*} \\
    
    Then $f = h^{_1}g: z_i \to w_i$ as required. \\
    \\
    Now to prove uniqueness\\
    Suppose $f': z_i \to w_i$.\\
    Then $f^{-1}f': z_i \to z_i$\\
    Let $g$ be as above. Can show $gf^{-1} f' g^{-1} = \text{id}$, \\
    so $f = f'$.
\end{answer}





%%%%%%%%%%%%%%%%%%%%%%%%%%%%%%%%%% LECTURE 24
\section* {Lecture 24}


\begin{question}
    Any non-identity Mobius map is conjugate to one of \\
    (i) $z \to \omega z$ \\
    (ii) $z \to z + 1$.
\end{question}
\begin{answer}
    Consider jordan normal forms
\end{answer}

\begin{question}
    Prove that a non-identity Mobius map has either 1 or 2 fixed points 
\end{question}
\begin{answer}
    Note fixed points are preserved under conjugation. Hence consider number of
    fixed points of $\omega z$ and $z + 1$
\end{answer}

\begin{question}
    If $f \in \mobius$ and $C$ is a circle, prove that $f(C)$ is a circle.
\end{question}
\begin{answer}
    It suffices to consider the transforms which constitute a mobius transform
    and show that these individually preserve circles. Consider the complex equation
    of a circle to verify each case.
\end{answer}

\begin{question}
    Define the cross-ratio of $z_1, z_2, z_3, z_4 \in \complex_\infty$.
\end{question}
\begin{answer}
        $$[z_1, z_2, z_3, z_4] = \frac{(z_1 - z_2)(z_2 - z_4)}{(z_1 - z_2)(z_3 - z_2)}$$
    If we have $\omega_i = \infty$, just apply the dubious simplifiation of $\frac{\infty}{\infty} = 1$.
\end{answer}

\begin{question}
    Given $z_1, z_2, z_3, z_4 \in \complex_\infty$
    and $\omega_1, \omega_2, \omega_3, \omega_4 \in \complex_\infty$,
    then $\exists f \in \mobius$ such that $f(z_i) = \omega_i$ iff 
    $[z_1, z_2, z_3, z_4] = [\omega_1, \omega_2, \omega_3, \omega_4]$
\end{question}
\begin{answer}
    $(\Rightarrow)$ Suppose $f(z_j) = \omega_j$. Find $\omega_j - \omega_k$ in terms of
    function and show cross-ratios are the same. \\

    $(\Leftarrow)$ Consider g mapping $z_i$ to $0,1,\infty$ and h mapping $\omega_i$ 
    to $0,1,\infty$, and using $[0,1,x,\infty] = x$, show that $h^{-1}g$ works.
\end{answer}

\begin{question}
    $z_1, z_2, z_3, z_4$ lie in some circle in $\complex_\infty$ iff
    $[z_1, z_2, z_3, z_4] \in \reals$.
\end{question}
\begin{answer}
    Construct a function
        $$: C \to \reals \cup \{ \infty \}$$
    and construct circle out of $z_1, z_2, z_4$. Map $g: z_i$ to $0,1,\infty$
    and apply cross-ratios to show that $z_3 \in C \iff [z_1, z_2, z_3,z_4] \in \reals$.
\end{answer}








\end{document}